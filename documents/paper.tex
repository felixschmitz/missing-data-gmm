\documentclass[11pt, a4paper, leqno]{article}
\usepackage{a4wide}
\usepackage[T1]{fontenc}
\usepackage[utf8]{inputenc}
\usepackage{float, afterpage, rotating, graphicx}
\usepackage{epstopdf}
\usepackage{longtable, booktabs, tabularx}
\usepackage{fancyvrb, moreverb, relsize}
\usepackage{eurosym, calc}
% \usepackage{chngcntr}
\usepackage{amsmath, amssymb, amsfonts, amsthm, bm}
\usepackage{caption}
\usepackage{mdwlist}
\usepackage{xfrac}
\usepackage{setspace}
\usepackage[dvipsnames]{xcolor}
\usepackage{subcaption}
\usepackage{minibox}
% \usepackage{pdf14} % Enable for Manuscriptcentral -- can't handle pdf 1.5
% \usepackage{endfloat} % Enable to move tables / figures to the end. Useful for some
% submissions.

\usepackage[unicode=true]{hyperref}
\hypersetup{
    colorlinks=true,
    linkcolor=black,
    anchorcolor=black,
    citecolor=NavyBlue,
    filecolor=black,
    menucolor=black,
    runcolor=black,
    urlcolor=NavyBlue
}


\widowpenalty=10000
\clubpenalty=10000

\setlength{\parskip}{1ex}
\setlength{\parindent}{0ex}
\setstretch{1.5}


\begin{document}

\title{Reproduction and Expansion of \\"A GMM Approach For Dealing With Missing Data On Regressors" by Jason Abrevaya and Stephen G. Donald}

\author{Florian Husch, Felix Schmitz, Timothy Voeste}

\date{
    {\bf Under construction}
    \\[1ex]
    \today
}

\maketitle
\footnotetext[1]{Email: \href{mailto:s3flhusc@uni-bonn.de}{s3flhusc@uni-bonn.de}, \href{mailto:s87fschm@uni-bonn.de}{s87fschm@uni-bonn.de}, \href{mailto:s6tivoes@uni-bonn.de}{s6tivoes@uni-bonn.de}}

\begin{abstract}
    Some abstract here.
\end{abstract}

\clearpage

\input{../bld/tables/simulation_results_design9.tex}

\input{../bld/tables/simulation_results_design10.tex}

\input{../bld/tables/simulation_results_design5.tex}

Table \ref{table:MCReplicationResultsDesign5} on page \pageref{table:MCReplicationResultsDesign5} refers
to the simulation presented in the paper, and Table \ref{table:MCReplicationResultsDesign1} on page \pageref{table:MCReplicationResultsDesign1}
refers to the first design in the appendix.

\newpage

\appendix
\section{Appendix: Reproduction of different designs}

Notice how the curvature in $beta_0$ increases with the amount of heteroskedasticity introduced.
Going from homoskedastic error terms in Figure \ref{fig:gamma20_homoskedastic} to heteroskedastic error terms dependent only on  imputation model level in Figure \ref{fig:gamma20_heteroskedastic_imputation}, to hetereoskedasticity also in the linear regression model in Figure \ref{fig:gamma20_heteroskedastic_regression}.

\begin{figure}
    \centering
    \includegraphics[width=0.8\textwidth]{../bld/figures/simulation_results_gamma20_homoskedastic.png}
    \caption{different $\gamma_{20}$ values, homoskedastic}
    \label{fig:gamma20_homoskedastic}
\end{figure}

\begin{figure}
    \centering
    \includegraphics[width=0.8\textwidth]{../bld/figures/simulation_results_gamma20_heteroskedastic_imputation.png}
    \caption{different $\gamma_{20}$ values, heteroskedastic imputation}
    \label{fig:gamma20_heteroskedastic_imputation}
\end{figure}

\begin{figure}
    \centering
    \includegraphics[width=0.8\textwidth]{../bld/figures/simulation_results_gamma20_heteroskedastic_regression.png}
    \caption{different $\gamma_{20}$ values, heteroskedastic regression}
    \label{fig:gamma20_heteroskedastic_regression}
\end{figure}

\iffalse
\input{../bld/tables/simulation_results_design1.tex}
\input{../bld/tables/simulation_results_design2.tex}
\input{../bld/tables/simulation_results_design3.tex}
\input{../bld/tables/simulation_results_design4.tex}
\input{../bld/tables/simulation_results_design5.tex}
\input{../bld/tables/simulation_results_design6.tex}
\input{../bld/tables/simulation_results_design7.tex}
\input{../bld/tables/simulation_results_design8.tex}
\fi


\end{document}
