\section{Introduction}
Missing data on regressors is a typical problem for empiricists.
Not only ignoring this constitution of the data can lead to biased and inconsistent estimates, but also using faulty and theoretically inadequate methods.
This paper aims to replicate the results of \cite{abrevaya2017}, which facilitate the Generalized Method of Moments to achieve efficiency gains upon commonly used methods on missing data by empiricists.
In addition to the replication, the aim is to study the behavior of the GMM approach in a variety of settings.
% such as different sample sizes, missing data mechanisms, and the number of regressors.
Further, the analysis is extended with a focus on the dummy variable imputation method, which is widely used in empirical research.
The goal is to provide a comprehensive comparison of the GMM approach and the dummy variable method in terms of bias, efficiency, and consistency, study the nature of the imputation model, and underline the relevance of applicable methods to achieve sound empirical recommendations.
The paper is structured as follows: Section 2 provides a literature review on the topic of missing data on regressors, Section 3 introduces the methodology of the GMM approach and the dummy variable method,
% further linking to the IV estimator,
Section 4 presents diverse and in-depth Monte Carlo simulations, Section 5 discusses an empirical application, and Section 6 concludes.
