\section{Conclusion}

This paper compares the GMM with the dummy variable method for handling missing data on regressors, emphasizing the role of the parameter $\gamma_{20}$ in determining the instrument's relevance. GMM utilizes moment conditions derived from model assumptions, ensuring consistent and efficient estimates even under complex missing data patterns. In contrast, the dummy variable method, while straightforward, introduces bias and inconsistency, particularly when the instrument $z_{2i}$ has explanatory power for the missing data mechanism in $x_i$.

The parameter $\gamma_{20}$ captures the extent to which the instrument $z_{2i}$ explains the variation in the potentially missing variable $x_i$. When $\gamma_{20} \neq 0$, the instrument provides relevant information for imputation, rendering the dummy variable method inappropriate due to omitted-variable bias. Conversely, GMM effectively incorporates this information, leveraging the instrument's relevance to correct for the endogenous selection process.

Monte Carlo simulations demonstrate that as $\gamma_{20}$ deviates from zero, the dummy variable method suffers from increasing bias, while GMM consistently delivers robust and efficient estimates across varying conditions, including homoskedastic and heteroskedastic residuals. These results highlight the critical importance of recognizing and appropriately modeling the relevance of $\gamma_{20}$ when selecting estimation methods for missing data problems.

This study's findings underscore the need for empirical researchers to evaluate the relevance of instruments carefully when addressing missing data, particularly in econometric models using panel data. The proposed GMM framework provides a rigorous approach to handling missing data in the presence of relevant instruments, enhancing estimation accuracy and validity.
Future research directions may explore the integration of this GMM approach with advanced data generation techniques, such as Wasserstein GANs and synthetic data models, especially in panel data settings (see e.g., \citet{athey2019, stanley2024}). Such a combination could preserve complex dependencies while further improving imputation accuracy, paving the way for more robust econometric analysis in the presence of missing data.
